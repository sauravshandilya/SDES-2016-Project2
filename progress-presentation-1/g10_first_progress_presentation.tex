\documentclass[10pt,handout,english]{beamer}
\usetheme{Berlin}
\usecolortheme{seahorse}
\usepackage{amsmath}

\usepackage{graphicx} %%For loading image files
\graphicspath{ {images/} } %folder-location for images

\title[] % (optional, only for long titles)
{Python API for Mobile Robot Control}
\subtitle{Progress Presentation-1}
\author[AE-663 Course Project ] % (optional, for multiple authors)
{Parin Chheda (153076005) \\ Saurav Shandilya (153076004) \\ Group-10 }
\institute [Indian Institute of Technology Bombay]% (optional)
{
  
}
\date[\today] % (optional)
{Prof. Prabhu Ramchandra \\ PRof. Madhu Belur \\ Prof. Kumar Appaiah}
%\subject{Computer Science}

\begin{document}
\frame{\titlepage}

\begin{frame}{Objective}

\end{frame}

\begin{frame}{System Architecture}

\includegraphics[scale = 0.35]{system_diagram}

\end{frame}

\begin{frame}{Milestone Achieved}
\begin{enumerate}
\item Robot-end firmware - written in embedded C
\item Serial Communication between Raspberry Pi and Robot
\item Function for configuring all IO Ports and Pins of microcontroller
\item Implemented and tested code for Buzzer and BarLED.  
\item Test code for Port and Pin configuration function
\end{enumerate}

\end{frame}

\begin{frame}{Future Work}
\begin{itemize}
\item Develop and test object-oriented implementation
\item Access following peripherals for ${\mu}$controller
	\begin{itemize}
	\item Timers
	\item ADC
	\item Interrupt
	\item I2C
	\end{itemize}
\item Improve data packet by incorporating checksum, end of packet payload to existing system.
\item Design PyQT GUI for making SFTP connection to Raspberry Pi from PC
\item Provide higher level abstraction for peripheral devices  
\end{itemize}

\end{frame}


\end{document}